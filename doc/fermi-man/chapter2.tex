\chapter*{Fermi Input Manual}

%%%%%%%%%%%%%%%%%%%%%%%%%%%%%%%%%%%%%%%%%%%%%%%%%%%%%%%%%%%%%%%%%%%%%%%%%%%%%%%%%%%%

\definecolor{mygreen}{rgb}{0.3, 0.9, 0.1}

%%%%%%%%%%%%%%%%%%%%%%%%%%%%%%%%%%%%%%%%%%%%%%%%%%%%%%%%%%%%%%%%%%%%%%%%%%%%%%%%%%%

\section{Input file}

\subsection{Mesh information}

\begin{Verbatim}[frame=single,commandchars=\\\{\}]
\textcolor{Red}{$mesh}

    \textcolor{OliveGreen}{mesh_file}    cube-multi-materials.msh
    \textcolor{OliveGreen}{dimension}    3
    \textcolor{OliveGreen}{parfile_e}    cube-multi-materials.epart
    \textcolor{OliveGreen}{parfile_n}    cube-multi-materials.npart

\textcolor{Red}{$end_mesh} 
\end{Verbatim}

\subsection{Energy groups information}

\begin{Verbatim}[frame=single,commandchars=\\\{\}]
\textcolor{Red}{$energy_groups}

    \textcolor{OliveGreen}{numenergy}  1
    \textcolor{OliveGreen}{numprecur}  2
    
\textcolor{Red}{$end_energy_groups}
\end{Verbatim}

\subsection{Calculation mode}

\begin{Verbatim}[frame=single,commandchars=\\\{\}]
\textcolor{Red}{$calculation_mode}

   \textcolor{OliveGreen}{static} ONLY_ONE 
\textcolor{Gray}{#    static PARAMETRIC {"0",XSA1={0.01,0.01,0.02}}, {"1",XSA1={0.01,0.01,0.02}} }
\textcolor{Gray}{#    transient tf=10.0 dt=0.5 }

\textcolor{Red}{$end_mode}
\end{Verbatim}

\subsection{Nuclear data information}

\begin{Verbatim}[frame=single,commandchars=\\\{\}]
\textcolor{Red}{$nuclear_data}

   \textcolor{OliveGreen}{xsfile}          xs.fermi
   
\textcolor{Red}{$end_nuclear_data}
\end{Verbatim}

%%%%%%%%%%%%%%%%%%%%%%%%%%%%%%%%%%%%%%%%%%%%%%%%%%%%%%%%%%%%%%%%%%%%%%%%%%%%%%%%%%%%

\section{Nuclear data}

\subsection{Macroscopic cross sections}


\begin{Verbatim}[frame=single,commandchars=\\\{\}]
\textcolor{Red}{$cross_sections}

     <material name> <0|1> <D> <XSa> <XSs> <nXSf> <eXSf> <chi>
     
\textcolor{Red}{$end_cross_sections}
\end{Verbatim}

\noindent
Where \verb <D> are the diffusion coefficients given as:

\begin{alltt}
<D> = < \(D_{1}\) \(D_{2} \dots D_{g}\) > 
\end{alltt}

\noindent
Where \verb <XSa> is the absortion cross sections given as:

\begin{alltt}
<XSa> = < \(\sigma_{1}^{a}\) \(\nu\sigma_{2}^{a} \dots \nu\sigma_{g}^{a}\) > 
\end{alltt}

\noindent
Where \verb <XSs> are the scattering cross sections given as:

\begin{alltt}
<XSs> = < \(\sigma_{2\rightarrow1}\) \(\sigma_{3\rightarrow1}  \dots \sigma_{{g\text{-}1}\rightarrow{g}}\) > 
\end{alltt}

\noindent
Where \verb <nXSf> are the number of neutrons emitted per fission time the fission cross sections given as:

\begin{alltt}
<nXSf> = <  \(\nu\sigma_{1}^{f}\) \(\nu\sigma_{2}^{f} \dots \nu\sigma_{g}^{f}\) > 
\end{alltt}

\noindent
Where \verb <eXSf> are the energy per fission times the fission cross sections given as:

\begin{alltt}
<eXSf> = <  \(e\sigma_{1}^{f}\) \(e\sigma_{2}^{f} \dots e\sigma_{g}^{f}\) > 
\end{alltt}

\noindent
Where \verb <chi> is the fission spectrum given as:

\begin{alltt}
<chi> = < \( \chi_{1} \) \(\chi_{2}\dots\chi_{g}\) > 
\end{alltt}

\subsection{Fission precursors' constants}

\begin{Verbatim}[frame=single,commandchars=\\\{\}]
\textcolor{Red}{$precursor_constants}

   <beta> <lambda> <chi>
   
\textcolor{Red}{$end_cross_sections}
\end{Verbatim}

\noindent
Where \verb <beta>  are the fission precursors yields given as:

\begin{alltt}
<beta> = < \(\beta_{1}\) \(\beta_{2} \dots \beta_{G}\) > 
\end{alltt}

\noindent
Where \verb <lambda>  are the precursors' decay constants given as:

\begin{alltt}
<beta> = < \(\lambda_{1}\) \(\lambda_{2} \dots \lambda_{G}\) > 
\end{alltt}

\noindent
Where \verb <chi>  are the precursors' neutron energy spectrum given as:

\begin{alltt}
<chi> = < \(\chi_{11} \chi_{21} \dots \chi_{G1} \chi_{12} \dots \chi_{Gg}\) > 
\end{alltt}