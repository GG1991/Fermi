\chapter{Coupling Fermi}

\section{Introduction}

There are complex problems, like termohydraulic-neutronic coupled problems, in which different codes can solve with detail differents equations of the global system.
For example, there are codes that are benchmarked in neutronic calculations and others that are specialized solving termohydraulics.
For these reason, when the equations are coupled it is neccesary the communication between them.

\section{Communication via standard MPI}

Standard MPI allows users to share data between different codes.
The strategy proposed is that the particular codes that solves some equations of the global system (like Fermi) are slaves of a master code.
All slaves are connected only to the master, and they don't share data between them.
The master code sends orders to the codes: to begin calculations in a particular time step, to restart the calculation, to abort, etc.
It also sends values of variables of interest and receives anothers.

The slaves can run in serial or in parallel mode.
If multiple processes of some slave are running, only the root process of each code stablishes the MPI communication with the master code.

\section{MPI implementation on Fermi}

The architecture mounted on Fermi described in this section allows the code to communicate with a master code in a succesfull way.
The coupling actions are called in 4 instances:
\begin{enumerate}
\item at the beginning of the program;
\item at the beginning of each local step after coupled time step;
\item at the end of each coupled time step;
\item at the end of the program.
\end{enumerate}

Once the code is running, if the switch $COUPLING$ is set to $1$ in the input of fermi, the following actions are called:

\subsection*{Coupling at instance 1: at the beginning of the program}
At this stage it is necessary to stablish the MPI connection between the root process of Fermi and the master code.
Also, it is necessary to check the consistency in data between them.
For these reason, Fermi receives important parameters like the initial time value, 
the amount of time steps, the amount of variables that are necessary to receive in each time step and the amount of variables that are necessary to send after calculations.
After that, Fermi compares this values with data laoded from input.
If it is necessary change the local values printing $WARNING$ advise.

Pseudocode of this instance:

\begin{Verbatim}[frame=single,commandchars=\\\{\}]

  \textcolor{Gray}{iCode is the code ID}
  
  print(" MPI connection with code: ", iCode)

  \textcolor{Gray}{Service Name is constructed in base of code ID iCode}
  
  Srvc_Name = "Coupling_C" + iCode
  print(" Looking for service: ", Srvc_Name)

  \textcolor{Gray}{Connection tries with master code via Coupling_Comm MPI communicator}
  
  nTries = 0
  while (nTries < 5)\{
    error = MPI_Lookup_name (Srvc_Name, MPI_INFO_NULL, Port_Name)
    if (error == 0) \{
      print(" Can't find service, trying again")
    \}
    nTries++
  \}
  if (error == 0)\{
    print(" Service found at port: ", Port_Name, " After"<< nTries+1<<" tries.")
    nTries = 0
  \}
  while (nTries < 5)\{
    print(" Connecting...")
    error = MPI_Comm_connect(Port_Name, MPI_INFO_NULL, 0, MPI_COMM_SELF, Coupling_Comm)
    if (error == 0)\{
      print(" Can't connect to service, re-trying")
    \}
    nTries = nTries + 1
  \}

  if (error != 0)\{
    print(" ERROR: "<<error<<" connecting to server.")
    exit()
  \}

  print(" ...connected.")

  \textcolor{Gray}{Receiving data from master}

  print(" Receiving general parameters ")  
  tag = 100
  int status
  
  error = MPI_Recv (&\textcolor{OliveGreen}{t_0}, 1, MPI_DOUBLE_PRECISION, 0, tag, Comp_Comm, &status)
  error = MPI_Recv (&\textcolor{OliveGreen}{N_t}, 1, MPI_INTEGER, 0, tag, Comp_Comm, &status)
  error = MPI_Recv (&\textcolor{OliveGreen}{N_input_var}, 1, MPI_INTEGER, 0, tag, Comp_Comm, &status)
  error = MPI_Recv (&\textcolor{OliveGreen}{N_output_var}, 1, MPI_INTEGER, 0, tag, Comp_Comm, &status)

  \textcolor{Gray}{Modifying master code, it is possible to add another lines here,}
  \textcolor{Gray}{in order to share other parameters.}    
    
  print(" General parameters received.")

  \textcolor{Gray}{Check consistency in data}

  checkConsistency(\textcolor{OliveGreen}{t_0, N_t, N_input_var, N_output_var})
  
  \textcolor{Gray}{Receiving control instruction:}
  \textcolor{Gray}{0: restart step / 1: continue / 2: abort}


  error = MPI_Recv (&\textcolor{OliveGreen}{order}, 1, MPI_DOUBLE_PRECISION, 0, tag, Comp_Comm, &status)

\end{Verbatim}


\subsection*{Coupling at instance 2: at the beginning of each local step after coupled time step}

At the beginning of the first time step, Fermi receives guesses to input variables from master code.
These guesses corresponds to $t_{coup_1}$, which is not necessary $t_{local_1}$ 
($\Delta t_{local}$ could be smaller than $\Delta t_{coup}$, but never greater).
For these reason, the input variables set at $t_{local_1}$ are the interpolation between the initial condition and the variables received.
Note that in case that $\Delta t_{coup}$ is equal to $\Delta t_{local}$, the interpolation results in the variables received.
$\Delta t_{local}$ could be used smaller than $\Delta t_{coup}$ 
in order to catch cinetics effects that have orders of magnitude of time smallers than time constants in termohydralics, for example,
or in order to achieve accurancy in numerical calculations.

So, after calculations in each coupled time step, output variables are sended to master (see stage 3).
Master code evaluates convergency in calculations and after that, sends and order.
This order is received by Fermi.
It could restart all the calculation of the coupled step (which means restart from last $t_{coup}$), 
continue with new $t_{local}$ or abort calculations.
Input coupled variables are received after that (at this stage), 
and corresponds to the actual $t_{coup}$ in case of restart (because it is necessary to recalculate the time step),
or to the next $t_{coup}$ in case of continue.

In transitory evolutions this strategy is clear.
If there is not a clear initial condition, it could be calculated from a quasi-stationary step.
In quasi-stationary evolutions, it has not sense to define $\Delta t_{local}$ different from $\Delta t_{coup}$.

Pseudocode of this instance:

\begin{Verbatim}[frame=single,commandchars=\\\{\}]

  \textcolor{Gray}{Check if it is coupled time step}
  if(mod(iStep-1, coupFrecuency)==0)\{
    coupledTimeStep = true
  \}
  else\{
    coupledTimeStep = false
  \}

  \textcolor{Gray}{Receiving data}
  if(coupledTimeStep)\{
    error = MPI_Recv (&\textcolor{OliveGreen}{input_var}, \textcolor{OliveGreen}{N_output_var},
    MPI_DOUBLE_PRECISION, 0, tag, Comp_Comm, &status)
  \}
  
  \textcolor{Gray}{Interpolate data}
  input_var_local_step = interpolate(input_var(coup_step_i-1), input_var(coup_step_i))

\end{Verbatim}


\subsection*{Coupling at instance 3: at the end of each coupled time step}

At the end of each coupled time step, it is necessary to send calculated variables of interest.
Also, Fermi has to wait to the master instruction, which could be: restart, continue or abort.

Pseudocode of this instance:

\begin{Verbatim}[frame=single,commandchars=\\\{\}]





\end{Verbatim}


\subsection*{Coupling at instance 4: at the end of the program}

Before end the program, it is necessary to finish connection with master.

Pseudocode of this instance:

\begin{Verbatim}[frame=single,commandchars=\\\{\}]

  \textcolor{Gray}{Disconnecting}
  error = MPI_Comm_disconnect(&(Comp_Comm))
  
\end{Verbatim}
